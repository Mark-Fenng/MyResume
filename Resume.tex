%%%%%%%%%%%%%%%%%%%%%%%%%%%%%%%%%%%%%%%%%
% Important note:
% This template requires the resume.cls file to be in the same directory as the
% .tex file. The resume.cls file provides the resume style used for structuring the
% document.
%
%%%%%%%%%%%%%%%%%%%%%%%%%%%%%%%%%%%%%%%%%

%----------------------------------------------------------------------------------------
%	PACKAGES AND OTHER DOCUMENT CONFIGURATIONS
%----------------------------------------------------------------------------------------

\documentclass{resume} % Use the custom resume.cls style

\usepackage[left=0.75in,top=0.6in,right=0.75in,bottom=0.6in]{geometry} % Document margins
\newcommand{\tab}[1]{\hspace{.2667\textwidth}\rlap{#1}}
\newcommand{\itab}[1]{\hspace{0em}\rlap{#1}}
\name{Feng Yun} % Your name
\address{YunFengHIT@outlook.com} % Your address
%\address{123 Pleasant Lane \\ City, State 12345} % Your secondary addess (optional)
% \address{(+country code) phone number\\ emailhere@email.com} % Your phone number and email

\begin{document}

%----------------------------------------------------------------------------------------
%	EDUCATION SECTION
%----------------------------------------------------------------------------------------

\begin{rSection}{Education}
{\bf Harbin Institute of Technology, China} \hfill {\em September 2016 - Present} 
    \begin{itemize}[itemsep=-0.3em]
        \item Bachelor of Engineering(Expected 2020) in Computer Science and Technology\hfill { GPA: 83/100 }
        \item Courses: Advanced C Language and Programming(97/100),Python Programming(98/100),Java Network Programming(91/100),Computer Networks(90.1/100),Computer System(87.2/100) etc.
    \end{itemize}
\end{rSection}
%----------------------------------------------------------------------------------------
%	PROJECTS SECTION
%----------------------------------------------------------------------------------------
\begin{rSection}{Projects}
{\bf  \href{https://github.com/EdgeTranslate/EdgeTranslate}{Edge Translate}}{ -- A Browser Extension} \hfill {\em July 2018- Present}
\begin{itemize}[itemsep=-0.3em]
    \item Designed and developed a browser extension called “Edge Translate” to help users to have better experience on foreign language website reading.
    \item Edge Translate integrates the Google Translate API within the webpage and one simple template render engine is used to update the page according to translation contents dynamically.
    \item Published on the \href{https://chrome.google.com/webstore/detail/edgetranslate/bocbaocobfecmglnmeaeppambideimao}{Chrome} / \href{https://addons.mozilla.org/zh-CN/firefox/addon/edge_translate/}{Firefox} / \href{https://appcenter.browser.qq.com/search/detail?key=edgetranslate&id=bocbaocobfecmglnmeaeppambideimao%20&title=edgetranslate}{QQ browser} web stores and gained over 20,000 weekly active users and Received a score of 4.7/5.0.
\end{itemize}

{\bf \href{https://github.com/Mark-Fenng/QingXueFrontEnd}{Qingxue} Website for an Education Company} \hfill {\em December 2017 -  April 2018}
\begin{itemize}[itemsep=-0.3em]
    \item Designed and built an education website to provide services including publishing, managing, reserving and purchasing courses.
    \item Front end pages are implemented with VUE framework and NPM.
    \item Designed a database which includes over 30 tables obeyed 3NF.
    \item Deployed the whole system on an Ali Cloud server using uWSGI as the application server and Nginx server as a reverse proxy, static resource loading and load balancing.
\end{itemize}

{\bf An AI Meme Creation Tool Kit} \hfill {\em January 2019 - June 2019}
\begin{itemize}[itemsep=-0.3em]
    \item Designed and developed a meme creation tool kit using AI image processing technology to help Wechat users to make memes conveniently.
    \item Applied several AI image processing technology including CPTN text detection, Image inpainting, character recognition based on CRNN, VGG16 model trained with ImageNet dataset.
    \item Wechat users can make their favorite memes easily with this tool kit without using professional image processing software like Photoshop.
    \item Participated in Wechat Mini Program Application Development Competition and achieved National Third Prize.
\end{itemize}
    
{\bf \href{https://github.com/Mark-Fenng/TV-Ratings}{TV Viewing Ratings Dataset Analysis}} \hfill {\em November 2018 - January 2019}
\begin{itemize}[itemsep=-0.3em]
    \item Used Spark to analyze a 10 GB dataset of TV viewing ratings for a local TV station to predict the viewing ratings for the new TV programs.
    \item Used Spark SQL to clean the data and made some statistical tests.
    \item Applied random forest algorithm to predict viewing ratings and applied a user-based collaborative filtering algorithm to recommend TV shows for TV viewers. 
\end{itemize}
\end{rSection}
%--------------------------------------------------------------------------------
%    INTERNSHIP SECTION
%-----------------------------------------------------------------------------------------------
\begin{rSection}{Internship}
    {\bf Luliang Cloud Computing Operation Company} \hfill {\em July 2019- August 2019}\\
    {Technical engineer assistant}
        \begin{itemize}[itemsep=-0.3em]
            \item Participated in the strategy design and execution phase of several domestic surveying and mapping projects. 
            \item Proposed some optimization ideas to improve the system performance.
        \end{itemize}
\end{rSection}
%----------------------------------------------------------------------------------------
%	AWARDS SECTION
%----------------------------------------------------------------------------------------
\begin{rSection}{Awards and Scholarships} 
    \begin{itemize}[leftmargin=0pt,itemsep=-0.3em]
        \item { 2019 Chinese University Computer Competition – WeChat Mini Program Application Development Competition , National Third Prize} \hfill{\em July 2019}
        \item { National College Students Mathematical Modeling Competition, Provincial First Prize} \hfill{\em December 2018}
        \item { Northeastern Three Provinces Mathematical Modeling League, First Prize}  \hfill {\em June 2018}
        \item { Lan Qiao Cup Java Software Development Contest, Provincial Second Prize}  \hfill {\em March 2019}
        \item { Lan Qiao Cup C++ Software Development Contest, Provincial Third Prize}  \hfill {\em April 2018}
    \end{itemize}
\end{rSection}
%----------------------------------------------------------------------------------------
%	SKILLS SECTION
%----------------------------------------------------------------------------------------
\begin{rSection}{Skills}
    \begin{itemize}[leftmargin=0pt,itemsep=-0.3em] 
        \item {\bf Software Programming Languages: }
        Java, Javascript,  C, Python, C\#,  Matlab
        \item {\bf Hardware Programming: }
        Verilog, VHDL
        \item {\bf Big data computing framework: }
        Hadoop, Spark,Pregel
        \item {\bf Others: }
        Web front end design and development (NPM, VUE), back end development(Spring boot, Django) and server deployment(proficient in Linux operation system)\\
        Mathematical modeling ability
    \end{itemize}
\end{rSection}
\end{document}----------------------------

